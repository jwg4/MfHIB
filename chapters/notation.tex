\chapter{Conventions of notation}
\label{notation}

\begin{itemize}
\item
The first place a new term is defined it is written like \term{this}. A term which will be defined later is written like \fterm{this}. Until we have seen a definition we will just use the term for its \term{na\"ive} meaning. This means that you should try and make sense of the term as you would if it was used in nonmathematical language. When you see a proper definition you can go backwards and see if the definition adds anything to the previous uses of the term.\footnote{Of course, all words that are not given a formal mathematical definition are used in their naive sense; without being able to use this language we wouldn't be able to start saying anything or even defining anything. The \fterm{notation} just indicates that we are hopefully going to give a definition at some later point.}
\item
The phrase ``if and only if'' is abbreviated by mathematicians to ``iff'', usually written as if it were a word in its own right. The phrase ``without loss of generality'' is often abbreviated ``w.l.o.g.'', we write it ``wlog'' as though it was not a word. The phrase ``in other words'' does not already have an abbreviation -- since we tend to overruse it we will sometimes write it ``iow''.
\end{itemize}
