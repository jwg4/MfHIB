\chapter{How many?}
\label{howmany}
\depends{counting}
\sent{0.01}
\james{I was just re reading Chapters 3 \& 4, and I thought I should drop you a line
to encourage the creation of a new set, which we will call; (5,6).\footnote{Note that James confuses the set $\{5,6\}$ with the sequence $(5,6)$.}

At present (5,6) does not exist and is simply an idea, but there is no reason
why (5,6) can not exist at some point in the future. Indeed if we look back at
at the set which we call (1,2), we can see that, although it was a set on its own,
it was followed not long after by (3,4). This being true we can see that the idea
of new set creation, shortly after the arrival of a previous set, already has a precedent.}



\begin{itemize}
\item
We have associated `numbers' with the `number of elements' in a given \fterm{set}.
\item
Clearly there is some circular reasoning here, since we don't know what it means for a set to have a given `number' of elements.
\item
All we have done is construct one list of sets, and associate each of these with a number. What we don't know how to do is to check if some other set has the same number of elements. If we did then we could say what the number of elements of a set it by finding the set among the original list of sets, and letting it be the number associated with that set.
\item
For example, we mentioned the set $\{0,1,2,3\}$, and said that it has `$4$', or `four' elements. So any set that has the same number of elements as this set, has $4$ elements.
\item
For this to work, we have to also be sure that a set can't have two different numbers of elements. So it wouldn't be possible for a set to have the same number of elements as $\{0,1,2\}$ \highlight{and} the same number of elements as $\{0,1,2,3,4,5\}$.
\item
We should bear in mind that we haven't formally said what a set is yet. We will try and observe some rules of how we think a set should behave at the moment, and later when we define a set more strictly we will try and make sure that we haven't contradicted our informal idea of a set.
\item
Take for example the set $\{\clubsuit, \diamondsuit, \heartsuit, \spadesuit\}$.
Our gut feeling is that it has the same number of elements as the set $\{0,1,2,3\}$, and does not have the same number of elements as $\{0,1\}$, $\{0,1,2,3,4,5,6,7,8,9,10,11,12\}$ or any other set in this family.
\item
We try and understand what it means for $\{\clubsuit, \diamondsuit, \heartsuit, \spadesuit\}$ to have a \highlight{different} number of elements that $\{0,1\}$. The concept we would like to make precise is \highlight{more}: the former has more elements than the latter.
\item
Suppose that we remove one element from each set, leaving us for example with $\{\clubsuit, \diamondsuit, \spadesuit\}$ and $\{0\}$. The property we are trying to describe, `more', stayed true, and we might have made the question slightly simpler by making the sets smaller. (In any case, we are unlikely to have made things worse.)
\item
Now remove one more element from each set. We get (among other possibilities) the set $\{\diamondsuit, \spadesuit\}$ and the set $\{\}$, the empty set. This suggests a possibility -- we could establish a rule that a set which has at least one member has more elements than the empty set.
\item
If we try and use this idea to check that $\{\clubsuit, \diamondsuit, \heartsuit, \spadesuit\}$ and $\{0,1,2,3\}$ have the same number of elements, we remove elements one at a time from each to get:
\begin{align*}
\{\clubsuit, \diamondsuit, \heartsuit\} &\text{and} \{0,1,3\} \\
\{\diamondsuit, \heartsuit\} &\text{and} \{1,3\} \\
\{\heartsuit\} &\text{and} \{1\} \\
\{\} &\text{and} \{\} \\
\end{align*}
Providing we are happy that the empty set has the same number of elements as itself, we seem to have found something that works.
\item
So our rules are:
\begin{enumerate}[I]
\item
A set with at least one element has more elements than the empty set.
\item
The empty set has the same number of elements as itself.
\item
If neither of the above rules is conclusive, remove one element from each set and check again.
\end{enumerate}
\item
One nice conclusion from this: every set has the same number of elements as itself. To see this, remove the same element from each set. The two sets are still the same. Do this until you get to the empty set. Since both sets are still the same, they are both empty. Hence they have the same number of elements, and so did the original two sets.
\item
Another nice fact is that a set has at least as many elements as its \fterm{subset}. To understand this fact we have to not only define what a subset is, but define what it means for one number to be at least as big as (or bigger than, or smaller than) another. 
\end{itemize}

\theendnotes
