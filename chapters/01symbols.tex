\chapter{The symbols $0$ and $1$}
\label{01symbols}
\begin{itemize}
\item
The symbol $1$ is natural, very old and may have been invented many times. It has its origin in the use of `tallies' to count by drawing lines or making notches in wood or bone. This way of recording numbers could more formally be referred to as \fterm{unary notation}.
\item
The invention of the symbol $0$ is something of a stroke of genius; not mathematical genius but design genius. It complements the symbol $1$ perfectly and sets itself up in opposition to it. 
\item
There are many important reasons why $1$ and $0$ make sense together.
\item
Since numerals for \fterm{two}, \fterm{three}, etc., particularly the Indian numerals, represent collections of $1$s, the symbol for zero had to be something which did not seem like it was made up of $1$s or short line strokes. The circle fits this requirement since no part of a circle is a straight line. The Indian dot for zero also satisfies this since a dot does not contain any line or piece of a line.
\item 
$1$ represents a movement, whereas $0$ represents no movement since the pen returns to its starting place. The movement back and round when drawing a $0$ cancels out the original, positive movement by adding an equal and opposite movement to it. See \fterm{vectors}. 
\item
The symbol $0$ shows a hole, a place where an object could be contained, and thus represents an absence. The symbol $1$ could fit inside this hole, just as the penis suggested by $1$ could fit inside the vagina suggested by $0$.
\item
The shapes $0$ and $1$ are reminiscent of the fundamental building blocks of geometry, the \fterm{circle} and the \fterm{line} or \fterm{line segment}. The two most important tools of geometry, the \fterm{compass} and the \fterm{ruler} also represent this dichotomy.
\item
$0$ is a \fterm{closed curve} and $1$ is an \fterm{open curve}. We can make a closed curve by adding something to an open curve, but we can never obtain a open curve from closed curves, no matter how many we add together.
\end{itemize}
