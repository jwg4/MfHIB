\chapter{How to read this book}
\label{howto}
\cquote{Perhaps this book will be understood only by someone who has himself already had the thoughts that are expressed in it--or at least similar thoughts.}{Wittgenstein, Tractatus Logico-Philosophicus}

\begin{itemize}
\item
If you don't understand something, you should almost never try and ``read through'' or skip ahead until you get to something you do understand. You should stop immediately and try to understand it. This could take, reasonably, anything between five minutes and three days. You might spend all this time thinking about the concept; or you might be looking for other explanations; someone to ask; related concepts which help you make sense of this one.
\item
The reason that it's much better to stop than to plough through has to do with the structure and language of mathematics. We learn a strategy for reading difficult prose, which however doesn't work for mathematics. When we read Shakespeare, we might encounter language that we don't understand. However, the concept that is being spoken about is almost certainly something we are familiar with; an emotion, an event, a character trait. By continuing to read, we manage to see the simple concept behind the difficult language, providing that we can understand enough of the words. Mathematics is the other way around - difficult, unheard-of concepts are explained using familiar, straightforward language. We should usually be able to puzzle things out eventually - but if we miss something out, we might never find out what that thing is.
\item
The second worst thing we can do is to convince ourselves that we have an understanding of something when we don't. All the concepts and processes that are built up on top of that thing will be nebulous. Imagine someone who has learnt to `count' by reciting numbers in order, but hasn't understood what the words refer to. That person is going to have difficulty learning to write numbers using digits and it will take them an enormous amout of work to add and subtract. Calculations beyond this will be so hard that they might never manage to do them properly.
\item
We have to ask ourselves at each point if we really understand a thing - can we give an example of it? Could we explain it to someone else? Asking these questions to ourselves is how we know that it's time to reflect on something tricky.
\item
This reflection time is the hardest part of learning mathematics.
\item
(The absolute worst thing we can do is get a complete, but false picture of what a concept is and how it works. But this is much less likely to happen, and is also much harder to guard against.)
\item
At the end of three or four days of thought you should definitely seek help; eg ask someone, including the author of the book if you know him.
\item
You probably shouldn't read more than two different chapters in any one day. If you find yourself doing this you are probably going too fast, or the book is full of things you already know. If you are going too fast; this means that you are either skipping things you don't understand, or that you are understanding things, but not spending enough time on them to really appreciate them.
\item
Mathematics has two purposes. It is the science of calculation. In studying what and how to calculate, you should be calculating yourself. It is also a tool of meditation. You should also be meditating on the mathematics you learn. This is no more or less important. 
\end{itemize}
