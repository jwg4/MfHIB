\chapter{Critique of the various probability theories}
\chaptermark{Critique of prob. theories}
\label{probcrit}
\begin{itemize}
\item
We examine all the theories of what probability means given in Chapter \ref{coin} in two ways. First we look at whether the definition of probability makes sense in the first place. Secondly we question whether the definition leads to the correct probability of $\frac{1}{2}$.
\item
The question we ask ourselves in each case ``Can we believe in this theory of probability?''
\begin{description}
\item[\term{Frequentist}]
The problem here is that knowledge about one small occurrence in the near future depends on knowledge about many other occurrences, some in the distant future. This means it could be very difficult to give probabilities for any events, and for many events it could be impossible. For example, the probability that a given horse will win the Grand National is equal to the long-term proportion of Grand Nationals it wins if many occur. If these are the actual, annually held Grand Nationals, then the probability of any horse winning the Grand National is zero or very close, as even the best horse will get old and stop winning, after which the proportion of wins will steadily decrease. If on the other hand the `long-term proportion' refers to multiple re-runs of this year's Grand National, with the same runners, conditions, etc., then it is clear that these must be imaginary. But if they are identical re-runs, shouldn't the same horse win every time? 
\item[\term{Multiple Universes}]
We have no way of knowing if other universes exist, and if they do we have no way of observing coin tosses there. Any decision about how many times a coin lands Heads in these universes, seems to in fact be based on arguments like the Symmetry one, the Outcome Space one, etc.
\item[\term{God Playing Dice}]
As with the Multiple Universes theory, this relies on something which is unobserved and unobservable. But even if we are willing to believe in the existence of God, it seems hard to understand how this strategy would work. Does God also decide what the probabilities of all events are? Or are they set down by a higher authority such as a symmetry law or God's God? 
\item[\term{Outcome space}]
This rests on the meaning of `equally likely'. In the case of the coin, and other similar cases, we seem to be able to justify this theory using the Symmetrical argument, a more advanced version of this one. In the case of more complicated events, it is not clear why the event happening and the event not happening are considered equally likely. At best we might end up with an oversimplified version of probability, where all events are ``fifty-fifty'' except those which are bound to happen and those which are impossible.
\item[\term{Symmetrical}]

\item[\term{Subjectivist}]

\item[\term{Pricing-theoretical}]
As with the Frequentist approach, we have to decide whether the entities referred to in our definition are real or idealized. We know that some real people will accept bets which are heavily weighted against them. On the other hand, some people would rather not take any risks with their money, even if they can theoretically make a decent profit. This depends among other things on the amount of money they have and how much is at stake in the bet. If we refer only to idealized gamblers/investors, then we need another theory to explain how they make decisions. 
\end{description}
\item
To decide if these theories offer a good rationale for the actual value $\frac{1}{2}$, we ask if another value for the probability of Heads could be consistent with the theory. We should bear in mind that there might be an easy way of showing that the probability of Heads is not $0.0003$ or $0.95$. The real test of a theory is whether we can be sure that the probability is not $\frac{4651}{9303}$ or $0.50001$.
\begin{description}
\item[\term{Frequentist}]
The probabilities given by the Frequentist approach are very approximate. A man called Jon Kerrich tossed a coin $10,000$ times while he was in a Nazi internment camp. He got $5,067$ heads. This is good evidence for the probability of Heads being $\frac{1}{2}$, but it is also perfectly decent evidence for the probability being $\frac{5067}{10000}$, $\frac{5032}{10000}$ or $\frac{4999}{10000}$.
\item[\term{Multiple Universes}]

\item[\term{God Playing Dice}]
This `calculation' is pretty much impossible in any reasonable interpretation. How can we count the multiple unseen universes which we do not inhabit, let alone determine how the coin falls in each of them? What about universes in which the coin is not tossed, or the universes in which the lack of gravity means that it stays suspended in the air? We could limit ourselves to including only those universes which we might possibly be in \emph{before} the coin is tossed, which exludes all the distant universes in which we do not exist or are not about to toss a coin. But this means we must also exclude either all the universes in which Lee Harvey Oswald assassinated JFK acting alone, or all the universes in which he was the patsy for a well-orchestrated conspiracy (or possibly both) since only one of the two sets of universes includes our universe. How can we do this if we don't know which set to exclude, and could this affect the probability?


\item[\term{Outcome space}]
This is a simple argument, and rests on the words ``equally likely''. 
\item[\term{Symmetrical}]
Same argument, except that now we appeal specifically to the \fterm{symmetry} of the coin: since both sides of the coin are the same, the actual physical events of landing on one side or another must have equal probabilities to one another.
\item[\term{Subjectivist}]

\item[\term{Pricing-theoretical}]
\end{description}

\end{itemize}
