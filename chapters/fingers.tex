\chapter{Finger counting}
\label{fingers}
\depends{ambiguity}

\begin{itemize}
\item
Suppose that you are counting on your fingers from 1 to 10. You are going to start with both your hands made into fists, and end up with all your fingers spread out. We are going to count in the order ``Left thumb to left little finger, right little finger to right thumb''. In other words if you have your palms facing you we are counting fingers from left to right. \footnote{There are various other ways of counting on fingers, even if we only consider counting to 10 on the fingers of both hands. For instance many people open their fingers in a different order and Japanese people usually count by starting with all their fingers open, and closing one into their hands for each count.}
\item
We don't count ``zero''.
\item
As we say ``one'', we open the left thumb. As we say ``two'' we open the left index, and so on. As we say ``ten'' we open the right thumb.
\item
You should actually do this now to make sure you agree this is the procedure.
\item
Now we want to count backwards, from 10 to 1. We start in the position we ended up in, with all the fingers of both hands open.
\item
The first number we are going to say is ``ten''. The first finger we are going to move is the right thumb.
\item
There are two ways of proceeding. 
\begin{enumerate}
\item
We can say ``ten'' simultaneously 
