\chapter{A to B}
\label{atob}

\begin{itemize}
\item
Suppose that we have two \term{points}, which we label $0$ and $1$.
\item
We use the word `points' rather than `numbers' or some other name, to indicate that we want to think of $0$ and $1$ as being located in \fterm{space}.
\item
The two points in space give rise to two more concepts; if we accept that there are two points with different locations, we can imagine moving from one point to the other. So we require two new concepts, the movement from $0$ to $1$ and the movement from $1$ to $0$. 
\item
If $0$ and $1$ were the same, moving from one to the other would have no meaning. However, we do in fact consider them to be different, otherwise we wouldn't speak of two different points with different names.
\item
Furthermore, we could consider that $0$ and $1$ are different points, but that there is no possibility of moving oneself or moving something else from one point to the other. However, this is unsatisfying, because in that case there is no connection at all between the two points.
\item
Therefore, to say something interesting about the two points in space, we assume that they are different and that one can travel, or move something, from one to the other.\footnote{Note that it is not necessary to accept this. It simply leads to more interesting structures than if we assume, or believe, either of the other two possibilities. This is of course a matter of opinion. You might also choose to believe that the subjectively interesting framework is also \highlight{inherently right}, but neither of these is a mathematical statement.}
\item
So we have two points, $0$ and $1$, and two `journeys' or `motions' -- the movement from $0$ to $1$ and the movement from $1$ to $0$. 
\item
To make it easier to refer to them, we will give ``the movement from $0$ to $1$'' the name $\overline{a}$, and ``the movement from $1$ to $0$'' the name $\overline{b}$. We are going to replace these names with other, more logical ones later.
\item  
One should never confuse $\overline{a}$ with either $0$ or $1$. They are different types of objects. ``The journey is not the destination'', and neither is it the point of departure. For one thing, the movement from $0$ to $1$ depends on both the points $0$ and $1$. If we changed the point $1$, the point $0$ might stay the same, but the movement from $0$ to $1$ would be different.
\item
What happens if we \fterm{add} together $\overline{a}$ and $\overline{b}$? Suppose we take $\overline{a}$ first and follow it with $\overline{b}$\ldots This is the journey from $0$ to $1$ followed by the journey from $1$ to $0$. We could interpret this as the ``journey from $0$ to $0$'', the journey that consists of starting at $0$ and not going anywhere.
\item
Here we are \highlight{abstracting}, or getting rid of, what is actually making these journeys. $0$ and $1$ might be points in the physical space we live in, and we might be moving ourselves, or some other object, between these two places. Or they might be points in some imaginary, theoretical, metaphysical or spiritual space, and the thing making the journey will be something that belongs to this space. However, regardless of what is moving, we can agree that the movements from $0$ to $1$ and back from $1$ to $0$ put together, make a journey that starts and ends at $0$. 
\item
We are also ignoring any \highlight{change} that might have taken place to the thing that moved. If you hitchhike from New York to San Francisco, you might change as a person, and hitchhiking back to New York will probably not change you back into the person you were before you made the trip. However, if we consider only your location, and not anything else about you, we can consider that you have stayed in the same place, or made a journey from New York to New York.


\end{itemize}
