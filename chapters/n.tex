\chapter{$\mathbb{N}$}
\label{n}
\depends{sets,counting,successor}
\sent{0.02}

\begin{itemize}
\item
The set of counting numbers $1, 2, 3, \ldots$ is so important it has its own name (and chapter). It is called $\mathbb{N}$.
\item
We also refer to this set as ``the set of natural numbers'' and its elements as ``the natural numbers''.
\item
$\mathbb{N}$ contains all the numbers which belong to the sequence of sets 
\[ \{\}, \{0\}, \{0, 1\}, \{0,1,2\}, \{0, 1,2,3\}, \ldots \]
in chapter \ref{counting}, except for $0$. It could be defined as the set which includes the \highlight{number of elements} of all sets of the sequence, except for the empty set.
\item
Similarly, in the language of chapter \ref{successor}, it contains all numbers except for $0$. It can be defined as the set which includes $S0$, includes the successor of every number it includes, and does not include anything which is not implied by those two rules.
\item
Note that though both of these definitions as well as the definition involving \highlight{three little dots}: 
\[ \{1, 2, 3, 4, \ldots \} \]
seem to leave us totally clear about what the elements of the set are, none of them conform to the two types of definition from chapter \ref{sets} (lists of elements do not include lists with three little dots). We will accept the set nonetheless. This is one of the sets which cannot be written as a list of all its elements. To write sets like that using the other method (base set plus condition), we would have to already have a base set we can use. But the base sets must start somewhere! We will accept this set; doing so allows us to define many more interesting sets for which this set is a base set. 
\item
We could give the set $0, 1, 2, \ldots$ a special name, however it is slightly more convenient (and much more conventional) to give $1, 2, 3, \ldots$ a special name.
\item
In chapter \ref{sets} we referred naively to a set called `numbers'. We can now consider this set to be $\mathbb{N}$. (In later chapters we will define different sets which we might consider to include all numbers.)
\end{itemize}
