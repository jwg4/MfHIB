\chapter{The Toss of a Coin}
\label{coin}
\begin{itemize}
\item
We toss a coin, which leads to one of two results. We call these results \term{Heads} (H) or \term{Tails} (T). \footnote{An immediate objection here is that neither of these results may take place. For example the coin might fail to land or be caught, or might land on its edge.  In the rest of this chapter we ignore this objection. There are at least two justifications for doing this:
\begin{enumerate}
\item
All possibilities where the result is not Heads or Tails are so unlikely that we can disregard them. This is extremely unsatisfactory, because we are concerned here exactly with what the probabilities/likelihoods of different events are. It means that our statements about the probabilities of Heads and Tails later in the chapter become approximate rather than exact. Also, in practise inconclusive results when tossing a coin are not especially unlikely, for instance when letting it fall onto uneven ground. I have experienced more than one situation when this occurred.
\item
The other justification is that if the result of a coin toss is not either Heads or Tails, we will toss the coin again, `counting' the result of that toss as the result. This means that the definition of a `toss' is now a sequence of actual throws, ending only when Heads or Tails is reached. It could be argued that this is just a hidden use of the previous, flawed, justification -- what if the cointoss is repeatedly inconclusive, and before we get a conclusive result the universe comes to an end? Although we might believe this to be unlikely, this is not the same as saying it will never happen. The only answer to this is that if you are tossing a coin to decide who will bat first in a cricket game, it repeatedly lands on its edge, and in the meantime an apocalyptic event starts to take place, mathematics may not be the most useful tool to deal with this situation. 
\end{enumerate}
This is not a pointless digression. Banks and insurance companies have been criticized for claiming that events which they believed to be extremely unlikely, when they occur are somehow outside of the original `game' of risks which they had agreed to participate in.}
\item
The \term{probability} that the result will be Heads is $\frac{1}{2}$.
\item
The probability that the result will be Tails is $\frac{1}{2}$.
\item
These two statements can be understood as meaning any one of many different things:
\begin{description}
\item[\term{Frequentist}]
If we were to toss the same coin a large number of times, the \fterm{proportion} of times we get the result H would be close to its probability $\frac{1}{2}$ and the proportion of times that we get the result T would also close to its probability $\frac{1}{2}$.
\item[\term{Multiple Universes}]
Many universes exist and we are in one of them. The number of universes in which the result is Heads is half of the total number of universes which exist.
\item[\term{God Playing Dice}]
God exists and she decides, each time a coin is tossed, whether it will land Heads or Tails. The probabilities are information God uses in her decision making procedure, in this case meaning that she will not `favor' either Heads or Tails.
\item[\term{Outcome space}]
There are two outcomes, and both are equally likely. Therefore the total amount of probability which is available, $1$, is shared out equally, giving $\frac{1}{2}$ to each outcome.
\item[\term{Symmetrical}]
Same argument, except that now we appeal specifically to the \fterm{symmetry} of the coin: since both sides of the coin are the same, the actual physical events of landing on one side or another must have equal probabilities to one another.
\item[\term{Subjectivist}]
Probability is a measure of our uncertainty as to the outcome of the coin toss. If we had some belief that Heads was more likely than Tails, we would assign it a probability greater than $\frac{1}{2}$, but no more than $1$. If we thought it was less likely, we would assign a number at least $0$ but less than $\frac{1}{2}$. Since we have no reason to believe either of these, we give it the probability $\frac{1}{2}$.
\item[\term{Pricing-theoretical}]
Suppose that I offered to pay you $\$1$ if the result of the toss were Heads, and nothing if it were Tails. The maximum price that you would be prepared to pay for this exciting investment opportunity is half a dollar, so the probability of Heads is no more than $\frac{1}{2}$. On the other hand, the least I am happy to accept is also half a dollar, therefore the probability of Heads is no less than $\frac{1}{2}$. So it is exactly $\frac{1}{2}$.
\end{description}
\end{itemize}
