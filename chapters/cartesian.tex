\chapter{Cartesian products}
\label{cartesian}
\depends{tuples, empty, union}

\begin{itemize}
\item
Take two sets, $A$ and $B$. The \term{Cartesian product of $A$ and $B$}, written $A \times B$, is the following set:
\[ \{(x,y) : x \in A, y \in B \} \]
\item
Example: if $A = \{1, 3, 5\}$ and $B = \{``j'', ``k'''\}$, then 
\[ A \times B = \{(1, ``j''), (3, ``j''), (5, ``j''), (1, ``k''), (3, ``k'') , (5, ``k'') \} \]
\item
We can form Cartesian products with singleton sets:
\[ \{2\} \times \{J. Edgar Hoover, Lee Harvey Oswald, William the Lionheart\} =  \{(2, J. Edgar Hoover), (2, Lee Harvey Oswald), (2, William the Lionheart)\} \]
(this is what we did in Chapter \ref{addition} where we first mentioned tuples), \[ \{4\} \times \{6\} = \{(4,6)\} \]
with the empty set:
\[ \{\} \times \{a, b, c, d, e, 14 \} = \{\} \]
\[ \emptyset \times \emptyset = \emptyset \]
and with infinite sets:
\[ \ldots \]
\item
We can form the Cartesian product of a set with itself:

Any overlap between the members of the set isn't discarded or ignored like with unions (see Chapter \ref{union}). All members of both sets are used for making pairs.
\item
$A \times B$ is not the same as $B \times A$. They are similar, but the order of the pairs is different:

\item
We can form the product of more than two sets at once. This is a set which contains not pairs but longer tuples:

This can become confusing: $A \times B \times C$ is not the same as $(A \times B) \times C$ or $A \times (B \times C)$.

Sometimes we don't care about the exact bracketing, but usually we should.
\end{itemize}
