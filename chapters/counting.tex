\chapter{Counting}
\label{counting}

\depends{0and1}
\sent{0.01}
\james{I enjoyed the first two chapters I feel like I will be able to master counting, although it seemed more complicated than the first time I learnt about it. I am looking forward to chapters three and four.}

\begin{itemize}
\item
The first number we mentioned in this book was not zero or one, but two:
\begin{quote}
``We start with \highlight{two} numbers, zero and one.''
\end{quote}
\item
Although $0$ and $1$ are the first numbers we think about, in fact we need another number to describe these numbers -- there are two of them.
\item
By considering that the \fterm{set} of the numbers $0$ and $1$ is described by the number $2$, we are led to consider what sets the numbers $0$ and $1$ describe. (Here we haven't made it formal what it means for a number to `describe' a set -- we are relying on our naive understanding of numbers.)
\item
The number $0$ describes no objects. It represents an absence or a lack of things. The set described by $0$ does not contain any items.
\item
This set is the \term{empty set}. Given any thing, object, idea or entity, it does not belong to the empty set. Thus the empty set has \highlight{zero} members.
\item
  We can easily think of what it means for something to be `one' or to be described by $1$. But saying precisely what it means is harder than it is to describe zero and the empty set.
\item
Suppose I describe a set by saying that it contains a specific physical object and nothing else. The number of members of this set is $1$. 
\item
Suppose the object is a coat. Are the sleeves of the coat in the set? No, because otherwise the set would contain at least three members, the coat, its right sleeve and its left sleeve. The sleeves of the coat are not the coat, although they are part of the coat.
\item
We also have to think about the same thing to be thinking about the same set. If I claim that a set contains `my coat', but when you read this you think of it as containing `your coat', then we are not thinking about the same set (unless we share a coat). Even if you think about a coat of mine, not the one I was thinking of, we are not thinking about the same set.
\item
Suppose that I think about the set which contains everyone who assassinated JFK and no-one else. I might think that this set contains one member, Lee Harvey Oswald. You are thinking about the set which contains only Lee Harvey Oswald. Are we thinking about the same set? I believe that we are, but you think that Fidel Castro (and no-one else) assassinated JFK. Thus for you the two sets are different. We can check this by both of us thinking about the set that contains Lee Harvey Oswald \highlight{and} everyone who assassinated JFK. I think this set has only one member, but you think it has two members.
\item
We can agree to define a set which contains a single item as follows: Given any item, it is either not in the set, or it is equal to that item. This pushes back the problem to deciding whether or not things are equal. 
\item
Suppose that I think that the set containing all of JFK's assassins contains Oswald and no-one else. You think it contains Jimmy Hoffa and no-one else. The sets are the same, only if Lee Harvey Oswald and Jimmy Hoffa are equal. We might for instance both believe them to be the same person, who sometimes used a pseudonym. Another possibility is that, when talking about what happened on the day of JFK's assassination, we consider them to be equal, since we both believe that that person fired from the Book Depository. We use that set when we talk about whatever that person did the rest of the day, referring to a single person, whose name we disagree on. If we were to talk about another day, when for example Oswald was in Cuba and Hoffa was in Washington working in the Teamsters Union building, then we no longer accept the two as equal.
\item
The problem of equality does not arise in any meaningful way when we have zero objects.\footnote{We might question whether things which don't exist, eg. the members of the empty set, are all equal to each other or all different. But this question shouldn't trouble us too much.} When we have one object, we need to be able to decide whether or not other things are equal to it. But if we can answer the question of equality, this allows us to build much bigger sets.
\item
Thus, the fact that $0$ and $1$ are not equal to each other, means that the set containing both of them, which we write as \[\{0, 1\},\] has two members.
\item
We also know that the set does not have $0$ members, since otherwise $0$ and $1$ would not exist, and it does not have $1$ member, since otherwise $0$ would be equal to $1$.
\item
Here we cannot \fterm{prove} that $0$ and $1$ \highlight{both exist}, and \highlight{are different from one another}. But we consider these to be our articles of faith, without which we would have nothing to say about $0$ and $1$.
\item
Therefore it has another number of members, which we call $2$. We have shown that $2$ is not equal to either $0$ or $1$.
\item
Thus the set \[\{0, 1, 2\}\] contains \term{three} members. 
\item
To show that $3$ is different from $2$, we need to prove that $\{0,1\}$ has a different number of members from $\{0,1,2\}$. To do this we will need to decide much more precisely what it means for two sets to have the same, or a different number of members, and therefore precisely what a number is.
\item
However, if we are happy that adding a new, different member to a set changes the number of its members, we can now obtain as many numbers as we like. Just look at a set, decide how many members it has, then add that number to the set. This works if we start with $\{0,1\}$ or even the empty set $\{\}$:
\[ \{\} \rightarrow \{0\} \rightarrow \{0,1\} \rightarrow \{0,1,2\} \rightarrow \{0,1,2,3\} \rightarrow \{0,1,2,3,4\}, \ldots \]
\item
It doesn't work if for example we start with the set $\{1,2\}$. The number of its elements is $2$. But adding $2$ to the set does not change the number of members (only \highlight{different} members are counted):
\[ \{1,2\} \rightarrow \{1,2\} \rightarrow \{1,2\} \rightarrow \{1,2\} \ldots \]
\item
Somewhere here we also have to be sure that adding members to a set never gives us a number we already have -- not just that $5$ is different from $4$, but that $5$ is different from \highlight{all the numbers we have seen before}. Otherwise we get into a loop like the $\{1,2\}$ loop.
\item
Here the \highlight{names} we use for the numbers, eg $2$, $3$ and so on, are arbitrary and we do not try to justify them mathematically. We have already called our first two numbers $0$ and $1$. We use the usual names for the other numbers because it helps the reader to understand what we are talking about. We could also use any other symbols 
\[ \{\} \rightarrow \{\copyright\} \rightarrow \{\copyright,\checkmark\} \rightarrow \{\copyright, \checkmark, \maltese\} \rightarrow \{\copyright, \checkmark, \maltese,7\} \rightarrow \{\copyright, \checkmark, \maltese,7,\bigtriangleup\}, \ldots \]
where $\maltese$ is defined to be the number of elements of the set $\{\copyright, \checkmark\}$,
or just use descriptive names
\begin{align*}
&\{\} \rightarrow \{the number of elements of \{\}\} \rightarrow \\
&\{the number of elements of \{\}, the number of elements of \{the number of elements of \{\}\}\} \ldots 
\end{align*}
\end{itemize}

\theendnotes
