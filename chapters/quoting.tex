\chapter{Quoting}
\label{quoting}
\depends{}

\begin{itemize}
\item
A piece of written text that appears between quotes (`like this'') does not have the same meaning as if it did not appear within quotes. 
\item
We have used this occasionally already, as usual without being formal about its exact meaning until we had time to.
\item
This is called the `use/mention' dichotomy. I \emph{use} the word ``Germany'' in the following sentence:
\begin{quote}
Germany is a country in central Europe.
\end{quote}
but I only \emph{mention} it in this one:
\begin{quote}
``Germany'' starts with the letter `G'.
\end{quote}
\item
If you think of words or expressions as \emph{names} which each point to or refer to some idea, then when we read something we replace each word by its \fterm{referent} before putting the ideas together in our heads. Quoting is like telling the reader not to `look up' the word in his head, but just to think about the actual word as an idea in itself.
\item
This is natural to do when we don't understand a word or phrase.
\begin{quote}
The biscuit tin had SCBOERJSDFPCGHIOWEBXIOSDFW written on its lid.
\end{quote}
When we do understand something, the natural thing to do is interpret it:
\begin{quote}
The biscuit tin had the name of the last Emperor written on its lid.
\end{quote}
and we have to use quoting to stop this process from happening:
\begin{quote}
The biscuit tin had ``the name of the last Emperor'' written on its lid.
\end{quote}
\item
\begin{quote}
``The Taj Mahal'' is not ``a monument in India''.
\end{quote}
\item
\begin{quote}
``This'' is not a complete sentence.
\end{quote}
\end{itemize}
