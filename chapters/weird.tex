\chapter{Weird ordinals}
\label{weird}
\depends{carrying,ambiguity,mappings}

\begin{itemize}
\item
In chapter \ref{carrying}, we constructed infinitely many numbers which `come after' infinity or $\omega$. Do these numbers have any validity or is their definition simply a game based on ignoring obvious facts about infinity?
\item
Those nummbers were \highlight{ordinals}. We have seen in chapter \ref{ambiguity} that ordinals and \highlight{cardinals} exist. We also saw that there is a correspondance between ordinals and cardinals, to the extent that it is possible to use numbers without understanding the two different types.
\item
For all natural numbers, there is a unique ordinal corresponding to each cardinal and vice versa. For the numbers which come after the end of the list 
\[ 1, 2, 3, 4, 5, \ldots \]
this correspondance breaks down.  
\item
The ordinal number $\omega$ is defined to be `the first number which comes after \highlight{all} of $1,2,3,\ldots$'. It corresponds to the cardinal $\infty$, defined as `the number of elements of the set $\mathbb{N}$'.
\item
The next ordinal number is $\omega + 1$, defined as the successor to $\omega$.
\item
This ordinal number \highlight{also} corresponds to the cardinal $\infty$.

\item
Suppose that we have a collection of boxes, one labelled with each of the natural numbers. We know that (at least) one of the boxes contains a million pounds. How do we find the money? We simply open box $1$, then if it is empty, open box $2$, and so on. Providing the money does exist, we will eventually find it.
\item
Now suppose that we have a collection of boxes, this time labelled with all of the whole numbers (the natural numbers plus their negative versions and zero). If we start as before by opening $1$, then $2$, then $3$ and so on, we might never find the money. It could be in box $0$ for instance. Because there are infinitely many positive whole numbers, we never get to $0$ and never find the money. 
\end{itemize} 
