\chapter{Tuples}
\label{tuples}
\depends{addition}

\begin{itemize}
\item
In chapter \ref{addition} we used the construction $(a, 1)$ and $(b, 2)$, to make mathematical objects that contained all the information from $a$ and $b$, and also a little more.
\item
We are going to recognise the existence of all such \term{pairs}. If $x$ and $y$ are two things that we are able to refer to\footnote{We are not addressing the problem of which things ``exist'' and don't exist here. Whatever collection of things you believe, (which should probably include `numbers' if you for this far) we can use to form pairs with.}, we will also refer to the pair $(x, y)$.
\item
The pair $(y,x)$ also exists and is different to the pair $(x,y)$.
\item
We can form the pair $(x,x)$, where $x$ is some object.
\item
The last two points show the difference between pairs and sets containing two objects. 

\end{itemize}
