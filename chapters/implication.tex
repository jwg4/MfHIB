\chapter{Implication}
\label{implication}
\depends{subsets, true}

\begin{itemize}
\item
We said in Chapter \ref{subsets} that ``$A$ is a subset of $B$'' means that
\[  x \in A \text{ for every } x \text{ for which } x \in B .\]
\item
We could express this as follows:
\[ \text{if } x \in B \text{ then } x \in A .\]
\item
Here \term{if} has a particular meaning used a mathematical word. To analyse this, we will consider the general form of such ``if/then'' sentences.
\[ \text{if } p \text{ then } q \]
We are using $p$ and $q$ to stand for any \fterm{statement}.\footnote{This is a process of \fterm{abstraction}, which we will probably use a lot before we come eventually to analyse it. If we reason about the `sentence' ``if $p$, then $q$'' (not actually a sentence at all -- only by filling in something reasonable for $p$ and $q$ does it become a sentence), we are actually saying something like this: Replace the symbol $p$ by some thing, and the symbol $q$ by some thing, where the things are chosen according to whatever guidelines we set down (here that $p$ and $q$ are both \fterm{statements}. Then whatever reasoning we came up with, will be true about this new modified object.}
\item
 A \term{statement} is something which can be either \fterm{True} or \fterm{False}.
\item
For example, we will distinguish between the sentence
\begin{quote}
The Earth orbits around the Sun.
\end{quote}
where we assume that if this sentence is not \emph{true}, then it is \emph{false}, and the sentence
\begin{quote}
Romeo, Romeo, wherefore art thou Romeo?
\end{quote}
where the fact that the sentence is not \emph{true}, does not mean that it is \emph{false}, and vice versa.\footnote{Note that it is quite easy to accept that a sentence is neither true nor false, but that we don't expect anyone to believe that a sentence can be \emph{both} true and false. This is convenient, because it means that although some things lie outside of True and False, everything that is part of the system must obey the rules of the system. We are prepared to tolerate \fterm{incompleteness}, as long as we don't have to put up with \fterm{inconsistency}. Similarly we might say that nothing can be two different colors at once, then anything which is say, red and green, we deal with by saying that anything that is two different colors actually does not have a color\ldots Something we might look at later is the ubiquity of this trade-off, the fact that every system is threatened by something which might break its rules -- to avoid the consequences of this we just say that that thing is outside the system of rules.}
\item
Some more examples of sentences which are either true or they are false:
\begin{itemize}
 \item
``I wish I was an only child like you, Wigger.''
\item
``You just put your lips together and blow.''
\item
``A rose is a rose is a rose is a rose.''
\end{itemize}
\item
And more sentences which need not be either true nor false:
\begin{itemize}
\item
``What did you do in the Great War, Daddy?''
\item
 ``Do not stand at my grave and weep.''
\item
``Nothing, nothing, nothing, nothing, nothing!''
\end{itemize}
\item
We will refer to the first category of sentences as \term{statements}.\footnote{What do we mean when we say something can be either true or false? Clearly many, if not all, statements cannot be \emph{either} true or false, they are just one of the two. If I \emph{know} that the statement
\begin{quote}
Lee Harvey Oswald assasinated John F. Kennedy, and he acted alone.
\end{quote}
is true, I might find it hard to describe it as `either true or false'. Whether we recognize uncertainty or not, we will accept all sentences which are \emph{true}, all sentences which are \emph{false}, and optionally some other statements which we cannot determine, into the world of `statements'.}
\item
When we use the sentence construction 
\[ \text{if } p \text{ then } q \]
and both $p$ and $q$ are statements, we obtain another statement. This new statement is either true or false, and which one it is depends only on the truth or falsity of $p$ and that of $q$.
\item
This is the first change from the everyday use of the terms `if' and `then'. `If' in usual language implies some causation between two facts. Since just by knowing whether $p$ is true or false and whether $q$ is true or false, we don't find out anything about how or whether $q$ depends on $p$, the truth of a sentence using `if' in real life does not \emph{only} depend on the truth of the inner statements.
\item
Examples:
\begin{quote}
If ``it rains tomorrow'', then ``we will go to the zoo''.
\end{quote}
\begin{quote}
If ``it didn't rain yesterday'', then ``Mao Tse-Tung killed 40 million people during the Cultural Revolution''.
\end{quote}
\end{itemize}
 
